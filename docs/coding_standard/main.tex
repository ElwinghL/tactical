\documentclass[12pt,a4paper]{article}

\usepackage[utf8]{inputenc}
\usepackage[english,french]{babel}
\usepackage[T1]{fontenc}

\usepackage{amsmath}
\usepackage{amsfonts}
\usepackage{amssymb}

\usepackage{url}

\usepackage{graphicx}

\usepackage{listings}
\usepackage{xcolor}

\lstset{literate=
  {á}{{\'a}}1 {é}{{\'e}}1 {í}{{\'i}}1 {ó}{{\'o}}1 {ú}{{\'u}}1
  {Á}{{\'A}}1 {É}{{\'E}}1 {Í}{{\'I}}1 {Ó}{{\'O}}1 {Ú}{{\'U}}1
  {à}{{\`a}}1 {è}{{\`e}}1 {ì}{{\`i}}1 {ò}{{\`o}}1 {ù}{{\`u}}1
  {À}{{\`A}}1 {È}{{\'E}}1 {Ì}{{\`I}}1 {Ò}{{\`O}}1 {Ù}{{\`U}}1
  {ä}{{\"a}}1 {ë}{{\"e}}1 {ï}{{\"i}}1 {ö}{{\"o}}1 {ü}{{\"u}}1
  {Ä}{{\"A}}1 {Ë}{{\"E}}1 {Ï}{{\"I}}1 {Ö}{{\"O}}1 {Ü}{{\"U}}1
  {â}{{\^a}}1 {ê}{{\^e}}1 {î}{{\^i}}1 {ô}{{\^o}}1 {û}{{\^u}}1
  {Â}{{\^A}}1 {Ê}{{\^E}}1 {Î}{{\^I}}1 {Ô}{{\^O}}1 {Û}{{\^U}}1
  {œ}{{\oe}}1 {Œ}{{\OE}}1 {æ}{{\ae}}1 {Æ}{{\AE}}1 {ß}{{\ss}}1
  {ű}{{\H{u}}}1 {Ű}{{\H{U}}}1 {ő}{{\H{o}}}1 {Ő}{{\H{O}}}1
  {ç}{{\c c}}1 {Ç}{{\c C}}1 {ø}{{\o}}1 {å}{{\r a}}1 {Å}{{\r A}}1
  {€}{{\euro}}1 {£}{{\pounds}}1 {«}{{\guillemotleft}}1
  {»}{{\guillemotright}}1 {ñ}{{\~n}}1 {Ñ}{{\~N}}1 {¿}{{?`}}1
}

\definecolor{commentcolor}{named}{lightgray}
\definecolor{numbercolor}{RGB}{152,204,112}
\definecolor{keywordcolor}{RGB}{60,128,49}
\definecolor{stringcolor}{named}{orange}

\lstset{language=C++,
  tabsize=4,
  basicstyle=\ttfamily,
  keywordstyle=\color{keywordcolor}\ttfamily\bfseries,
  stringstyle=\color{stringcolor}\ttfamily,
  numberstyle=\color{numbercolor}\ttfamily,
  commentstyle=\color{commentcolor}\itshape\ttfamily
}

\lstset{language=[11]C++}

\newcommand{\flenglish}[1]{\emph{\foreignlanguage{english}{#1}}}

\author{Fabien \textsc{Matusalem}}
\title{Projet tutoré : jeu \flenglish{tactical} \\ \flenglish{Coding Standard}}
\date{}

\begin{document}
\maketitle

Ce document récapitule les conventions d'écriture du code du projet. Il est nécessaire de les respecter\footnote{Sous peine de bras ou autres membres arrachés} afin de faciliter la lecture du code par n'importe quel membre du groupe. Toute proposition de modification est évidemment bienvenue.

Le style utilisé est le style \emph{K\&R} (style traditionnel du C), agrémenté de quelques règles provenant du style \emph{Stroustrup} pour convenir au C++.

Les différents opérateurs seront séparés d'une espace avec le reste de l'expression, selon les modèles ci-dessous.

\section{Règles de nommage}
Il a été décidé d'utiliser le \texttt{camelCase} pour les différents identifiants (variables, fonctions, classes, etc.), puisque c'est également le style du \flenglish{Gamedev Framework}.
Seuls les \lstinline|class|es (ou \flenglish{templates} de classes), \lstinline|struct|s, \lstinline|union|s, \lstinline|enum|s et constantes d'énumération auront des noms commençant par une majuscule.
Si un sigle ou un acronyme est utilisé, il est considéré comme tout autre mot : tout en minuscule, seule la première lettre est en majuscule sauf si c'est le premier mot d'un identificateur de variable, de fonction, etc.

\begin{lstlisting}
class MyClass {};

enum MyEnum {
	FirstValue,
	SecondValue,
	ThirdValue
};

class HtmlParser {};

HtmlParser htmlParser, otherHtmlParser;
\end{lstlisting}

\section{Code source}

Les codes sources suivants montre certaines conventions au niveau des espacements, indentations et saut de ligne. Pour plus de clarté, les espaces sont montrés (avec le caractère \og\textvisiblespace\fg).

Les fichiers sources auront pour extension \texttt{.cpp}.

\subsection{Définitions de variables, appels de fonction}
Notez que, contrairement au C, les caractères \og * \fg{} et \og \& \fg{} sont collés au type, plutôt qu'à l'identificateur. Évitez les déclarations de plusieurs pointeurs (et, généralement, de variables) sur une seule ligne.

\begin{lstlisting}[breaklines]
double* p {makePointer("blabla", 1337, {'a', 'b'})};
int i {-1};
int& j {i};
int* ptr = &j;

std::map<char, std::pair<int, int>> myMap {{'z', {0, 0}}, {'H', {4, 2}}};

std::vector<int> myVector {4}; // 1 élément: 4
std::vector<int> otherVector(4); // 4 éléments
\end{lstlisting}

\subsection{Contrôle de flux}

Contrairement aux fonctions (cf. §\ref{sec:fct}), les mots-clefs des structures de contrôle sont suivis d'une espace avant une parenthèse ouvrante. L'accolade ouvrant le bloc est située en fin de ligne, après une espace.

\begin{lstlisting}
for (int i = 0; i < nElementCount; ++i) {
	int* p = nullptr;
	if (1 == i) {
		// Do something
	} else {
		// ...
	}
}

do {
	switch (someCharacter) {
	case 'A': {
			uselessFunction();
		}
		break;
	// ...
	default: {
			std::cerr << "error" << std::endl;
		}
	}

	// ...
} while (answer != 42 && !earthDestroyed);
\end{lstlisting}

\subsection{Définition de fonctions}
\label{sec:fct}

Les fonctions sont le \emph{seul} cas où l'accolade est seule sur une ligne. Cela permet une présentation plus propre, surtout au niveau de la liste d'initialisation des constructeurs.

Lors d'une définition ou d'un appel de fonction, son nom est collé à la parenthèse ouvrante.

\begin{lstlisting}
namespace myNamespace {
	std::string myFunction()
	{
		return "Hello world!";
	}
	
	void MyClass::MyClass(int number) :
		_number {number},
		_stringAttribute {"abcde"}
	{
		// Nothing
	}
} // namespace myNamespace
\end{lstlisting}

\section{Fichiers d'entête}
Les définitions se feront dans des fichiers \texttt{.h}. Les attributs d'une classe suivent la syntaxe \texttt{m\_}<\emph{nom}>.

\medskip
Rappelons qu'il faut protéger les fichiers d'entête par la \textbf{garde d'inclusion} et qu'\textbf{il ne faut pas y écrire d'implantation} (sauf \flenglish{templates} et fonctions \lstinline|inline|).
\begin{lstlisting}
// MyClass.hpp

#ifndef MYCLASS_HPP
#define MYCLASS_HPP

#include "MyOtherClass.h"
#include "ParentClass.h"

#include <string>

class MyClass : public ParentClass {
public:
	int getElementCount(void);

private:
	bool m_dead;
	int m_element;
	char* m_charList;
	std::string m_string;
	
	MyOtherClass* m_otherClass;
};

#endif //MYCLASS_HPP
\end{lstlisting}

\section{Indications et bonnes pratiques}
Cette section est un peu en dehors du \flenglish{coding standard}, mais certaines règles méritent d'être rappelées. La plupart sont inspirées des \flenglish{C++ guidelines} \footnote{\url{https://github.com/isocpp/CppCoreGuidelines}}.

Ces règles ne sont pas à respecter au pied de la lettre dans tous les cas. Vous pouvez décider de ne pas suivre une règle s'il y a une bonne raison : simplicité, lisibilité, performance, etc.

\begin{itemize}
\item Préférez les implantations utilisant les vérifications de type. Notamment, préférez :
	\begin{itemize}
	\item les fonctions \lstinline|constexpr| aux macros ;
	\item \lstinline|enum class| à \lstinline|enum| ;
	\item les \flenglish{templates} variadiques pour les fonctions variadiques.
	\end{itemize}
\item Utilisez les bibliothèques au lieu de refaire une fonctionnalité \og maison \fg{} : celles-ci sont souvent mieux codés et testés qu'un code fait en cinq minutes\dots{}
\item Bannissez le code \og à la C \fg{} en faveur du C++ moderne, par exemple, servez-vous de :
	\begin{itemize}
	\item \lstinline|std::shared_ptr| ou \lstinline|std::unique_ptr| pour gérer la mémoire dynamique ;
	\item références à la place des pointeurs, quand c'est possible ;
	\item \lstinline|std::array| plutôt que les tableaux intégrés au langage ;
	\item \lstinline|std::function| au lieu des pointeurs de fonction ;
	\item \flenglish{casts} (ou conversion de types) du C++ et pas du C.
	\end{itemize}
\item Rendez constant tout ce qui peut l'être avec \lstinline|const| ou \lstinline|constexpr| : le compilateur réalise une optimisation ou donne une erreur si ce qui est censé être constant est modifié.
\item Tout ce qui peut ne pas être copié ne doit pas l'être. Utilisez le déplacement (avec \lstinline|std::move|) ou le passage par référence constante. Sauf pour les \og petits \fg{} types, pour garder un code simple.
\item Si une classe est vouée à être héritée :
	\begin{itemize}
	\item déclarer le destructeur avec \lstinline|virtual| (même vide !) ;
	\item utiliser \lstinline|virtual| pour les méthodes virtuelle.
	\item supprimer la copie.
	\end{itemize}
\item Si une classe hérite d'une autre, spécifier soit \lstinline|override|, soit \lstinline|final| pour les méthodes héritées.
\item Un objet doit toujours être valide. C'est pour cette raison que les attributs d'une classe doivent toujours être privés. Oui, avec \lstinline|private| pas \lstinline|protected|. Si le constructeur ou une autre méthode échoue à garder l'objet valide, l'objet invalide ne doit pas pouvoir être utilisé.
\item La création d'objets globaux non-constants doit être utilisée avec parcimonie.
\item Ne vous répétez pas. Vous pouvez :
	\begin{itemize}
	\item factoriser du code dans des fonctions ;
	\item remplacer les \og constantes magiques \fg{} par de vraies constantes ;
	\item utiliser \lstinline|auto| quand le type utilisé est clair.
	\end{itemize}
\end{itemize}

\end{document}